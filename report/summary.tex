\chapter{Summary}
\label{cha:summary}

We have seen that earning fundamental dynamics in networks is not a trivial problem. Even on the simplest network topologies, different interactions of traffic arising from multiple sources, can create complex patterns in the overall traffic. However, amidst these complexities, there exist some underlying patterns, which can effectively be captured and learnt, by having the right model equipped for this task. Over a span of time, measurements taken on these traffic complexities, show that not all is truly random, there is some structure which can be learnt, and be doing so, this can be leveraged to improve performance in networks on a whole.
 
We present in this thesis, a new NTT model, which serves the first steps towards learning fundamental behaviour of dynamics from network traffic. Based on state-of-the-art techniques on learning sequences in data in other fields, we present our Transformer based architecture which is trained to learn similar sequential information in network traffic data. Of course, doing so even in our setup, is quite challenging and there is clear scope for a huge amount of improvement. We feel that this approach surely opens up a plethora of new research questions and directions in the field of learning for networks. Over the course of this project, we explore several possible methods of pre-training on traffic traces, following which we evaluate and compare them, in order to understand better, the possibilities and limits of learning the network dynamics. Based on our findings, we do conclude that learning these networks dynamics is definitely possible, and acknowledge the fact that a lot is still unknown about deep learning on such data, but at the same time, realise that we are in a much better position to decide new directions to proceed in, given our current NTT architecture.
 
 We hope that our initial work in this direction, motivates the networking community to explore the vast possibilities of taking a step further in this domain and working together to build smarter and better learnt models, to improve performance and efficiency in the networks of tomorrow.